\RequirePackage{../recipe}
\begin{document}
%%%%%%%%%%%%%%%%%%%%%%%%%%%%%%%%%%%%

%%%%%%% Recipe Formatting %%%%%%%

% add your formatting for the first page here
\fancypagestyle{firstpage}{%
    % here is an example
    \noredmeat
    \prepcooktime{1 hour}{1 hour 45 minutes}
    \dinner
    \serves{6}
}

%%%%%%%%%%%%%%%%%%%%%%%%%%%%%%%%%
% start the recipe with the \recipe command
\recipe{Recipe Template}
% set the formatting for the first page only
\thispagestyle{firstpage}
% sign every page with name and url
\sign{Patrick Cook}{https://github.com/pdcook/Recipes}

% begin writing the recipe
\ingredients
\begin{itemize}
    \item 1 Recipe Template
    \item 1 Internet Connection
    \item 1 Text Editor (Preferably Vim)
    \item Some experience with \LaTeX
\end{itemize}

\instructions
Preheat computer to \temp{120}. Clone this repository or download the template file.
\begin{enumerate}
    \item Read through this template and figure out which of the few commands you will need.
    \item Edit template to taste.
    \item Use template to write down recipes. If in need of inspiration, I suggest \foothref{https://github.com/pdcook/Recipes/}{some of the recipes here.}
    \item Put in \LaTeX compiler of your choice, I use \texttt{pdflatex}. Compile until light golden brown.
    \item Serve immediately.
\end{enumerate}

\spewfootnotes

\end{document}
