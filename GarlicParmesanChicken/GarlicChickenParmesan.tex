\documentclass{article}
\usepackage{fancyhdr}
\usepackage{multicol}
\usepackage[hidelinks]{hyperref}
\usepackage[bottom]{footmisc}
\usepackage{lastpage}

% some default formatting
\pagestyle{fancy}
\cfoot{\thepage\ of \pageref{LastPage}}
\setlength\parindent{0pt}

%%%%%%% Commands for recipes %%%%%%%

% preceed recipe with \recipe{NAME} command
\newcommand{\recipe}[1]{%
    \newpage\lhead{}\chead{#1}\rhead{}\lfoot{}\rfoot{}\section*{#1}}

% display how many servings it makes with \serves{NUMBER}
\newcommand{\serves}[1]{%
    \chead{Serves #1}}

% options for different diets: \vegetarian, \vegan, \pescetarian, \noredmeat
\newcommand{\vegetarian}{%
    \rhead{Vegetarian}}
\newcommand{\vegan}{%
    \rhead{Vegan}}
\newcommand{\pescetarian}{%
    \rhead{Pescetarian}}
\newcommand{\noredmeat}{%
    \rhead{No Red Meat}}

% options for recipe type: \breakfast, \lunch, \dinner, \snack, \dessert
\newcommand{\breakfast}{%
    \lhead{Breakfast}}
\newcommand{\lunch}{%
    \lhead{Lunch}}
\newcommand{\dinner}{%
    \lhead{Dinner}}
\newcommand{\snack}{%
    \lhead{Snack}}
\newcommand{\Dessert}{%
    \lhead{Dessert}}

% display only one of preptime or cooktime with \preptime{PREPTIME} or \cooktime{COOKTIME}
% display both preptime and cooktime with \prepcooktime{PREPTIME}{COOKTIME}
\newcommand{\preptime}[1]{%
    \lfoot{Prep time: #1}}
\newcommand{\cooktime}[1]{%
    \lfoot{Cook time: #1}}
\newcommand{\prepcooktime}[2]{%
    \lfoot{Prep time: #1\\Cook time: #2}}

% start ingredients list with \ingredients or \ingredients[HEADER]
\newcommand{\ingredients}[1][\Large\emph{Ingredients}]{%
    \emph{#1}\\}
% start instructions list with \instructions or \instructions[HEADER]
\newcommand{\instructions}[1][\Large\emph{Instructions}]{%
    \emph{#1}\\}

% temperature in farenheit with \temp{TEMPERATURE}
\newcommand{\temp}[1]{%
    $#1^\circ$F}

% sign recipe with \sign{NAME}{URL}
\newcommand{\sign}[2]{%
    \rfoot{#1\\\emph{\href{#2}{#2}}\\}}

%%%%%%%%%%%%%%%%%%%%%%%%%%%%%%%%%%%%

%%%%%%% Recipe Formatting %%%%%%%

% add your formatting for the first page here
\fancypagestyle{firstpage}{%
    % here is an example
    \noredmeat
    \prepcooktime{30 minutes}{8 hours}
    \dinner
    \serves{6}
}

%%%%%%%%%%%%%%%%%%%%%%%%%%%%%%%%%

\begin{document}
% start the recipe with the \recipe command
\recipe{Garlic Chicken Parmesan}
% set the formatting for the first page only
\thispagestyle{firstpage}
% sign every page with name and url
\sign{Patrick Cook}{https://github.com/pdcook/Recipes}

% begin writing the recipe
\ingredients
\begin{multicols}{2}
\begin{itemize}
    \item 3 lb Chicken Breast
    \item 3 Baby Red Potatoes
    \item 10 Cloves Garlic
    \item 4 tbsp Irish Butter
    \item 4 tbsp Fresh Thyme
    \item 4 tbsp Fresh Chopped Parsley
    \item Grated Parmesan
    \item Olive Oil
\end{itemize}
\columnbreak
\end{multicols}

\instructions
Procure a medium-large crock-pot and a medium to large skillet.
\begin{enumerate}
    \item Cut the chicken into pieces smaller than a fist, then season the pieces with salt and pepper.
    \item Heat 1 to 2 tbsp of olive oil in the skillet on medium-high heat.
    \item Sear the chicken on all sides until golden brown. About 3 minutes per side.
    \item Wash and quarter the potatoes. Chop the garlic and parsely. Pull the leaves from the thyme. Add all to crock-pot.
    \item Add the chicken, butter, 4 heaping tbsp of grated parmesan, and 4 tbsp of oil to the crock-pot and season generously with salt and pepper. Mix everything together.
    \item Set the crock-pot pot to low and cook for 8 hours, occasionally stirring. Alternatively, set to high for 4 hours.
    \item Garnish with parmesan to taste and serve.
\end{enumerate}

\end{document}
