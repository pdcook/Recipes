\documentclass{article}
\usepackage{fancyhdr}
\usepackage{multicol}
\usepackage[hidelinks]{hyperref}
\usepackage[bottom]{footmisc}
\usepackage{lastpage}

% some default formatting
\pagestyle{fancy}
\cfoot{\thepage\ of \pageref{LastPage}}
\setlength\parindent{0pt}

% some custom commands for hyperlinks
\usepackage{tablefootnote}
\usepackage{dblfnote}
\hypersetup{%
    colorlinks=true,
    urlcolor=blue,
    linkcolor=blue,
    runcolor=blue,
    allcolors=blue}
\newcommand{\foothref}[2]{\href{#1}{#2}\tablefootnote{\mbox{\url{#1}}}}

\makeatletter
\newcommand{\spewfootnotes}{%
\tfn@tablefootnoteprintout%
\global\let\tfn@tablefootnoteprintout\relax%
\gdef\tfn@fnt{0}%
}
\makeatother

\makeatletter
\renewcommand\footnoterule{%
  \kern-3\p@
  \hrule\@width \textwidth
  \kern2.6\p@}
\makeatother

%%%%%%% Commands for recipes %%%%%%%

% preceed recipe with \recipe{NAME} command
\newcommand{\recipe}[1]{%
    \newpage\lhead{}\chead{#1}\rhead{}\lfoot{}\rfoot{}\section*{#1}}

% display how many servings it makes with \serves{NUMBER}
\newcommand{\serves}[1]{%
    \chead{Serves #1}}

% options for different diets: \vegetarian, \vegan, \pescetarian, \noredmeat
\newcommand{\vegetarian}{%
    \rhead{Vegetarian}}
\newcommand{\vegan}{%
    \rhead{Vegan}}
\newcommand{\pescetarian}{%
    \rhead{Pescetarian}}
\newcommand{\noredmeat}{%
    \rhead{No Red Meat}}

% options for recipe type: \breakfast, \lunch, \dinner, \snack, \dessert
\newcommand{\breakfast}{%
    \lhead{Breakfast}}
\newcommand{\lunch}{%
    \lhead{Lunch}}
\newcommand{\dinner}{%
    \lhead{Dinner}}
\newcommand{\snack}{%
    \lhead{Snack}}
\newcommand{\Dessert}{%
    \lhead{Dessert}}

% display only one of preptime or cooktime with \preptime{PREPTIME} or \cooktime{COOKTIME}
% display both preptime and cooktime with \prepcooktime{PREPTIME}{COOKTIME}
\newcommand{\preptime}[1]{%
    \lfoot{Prep time: #1}}
\newcommand{\cooktime}[1]{%
    \lfoot{Cook time: #1}}
\newcommand{\prepcooktime}[2]{%
    \lfoot{Prep time: #1\\Cook time: #2}}

% start ingredients list with \ingredients or \ingredients[HEADER]
\newcommand{\ingredients}[1][\Large\emph{Ingredients}]{%
    \emph{#1}\\}
% start instructions list with \instructions or \instructions[HEADER]
\newcommand{\instructions}[1][\Large\emph{Instructions}]{%
    \emph{#1}\\}

% temperature in farenheit with \temp{TEMPERATURE}
\newcommand{\temp}[1]{%
    $#1^\circ$F}

% sign recipe with \sign{NAME}{URL}
\newcommand{\sign}[2]{%
    \rfoot{#1\\\emph{\href{#2}{#2}}\\}}

%%%%%%%%%%%%%%%%%%%%%%%%%%%%%%%%%%%%

%%%%%%% Recipe Formatting %%%%%%%

% add your formatting for the first page here
\fancypagestyle{firstpage}{%
    % here is an example
    \vegetarian
    \prepcooktime{30 minutes}{8 hours}
    \dinner
    \serves{Many}
}

%%%%%%%%%%%%%%%%%%%%%%%%%%%%%%%%%

\begin{document}
% start the recipe with the \recipe command
\recipe{Meatless Beef Chili}
% set the formatting for the first page only
\thispagestyle{firstpage}
% sign every page with name and url
\sign{Patrick Cook}{https://github.com/pdcook/Recipes}



% begin writing the recipe
\ingredients
\begin{itemize}
    \renewcommand{\thefootnote}{\arabic{footnote}}
    \item $\sim 2$~lbs \foothref{https://www.quorn.us/products/quorn-meatless-grounds}{Quorn Meatless Grounds}\renewcommand{\thefootnote}{\fnsymbol{footnote}}\footnote[2]{\hbox{Optionally, use regular beef grounds if desired.}}
    \item 2 $15$~oz Cans of Dark Red Kidney Beans, Drained
    \item 2 $15$~oz Cans of Tomato Sauce
    \item 1 $28$~oz Can of Crushed Tomatoes
    \item 1 Small White Onion, Diced
    \renewcommand{\thefootnote}{\arabic{footnote}}
    \item 1 Box \foothref{https://carrollshelbyschili.com/products}{Carroll Shelby's Chili Kit}\renewcommand{\thefootnote}{\fnsymbol{footnote}}\footnote[3]{\hbox{If this exact box cannot be obtained, the contents are:} \hbox{$32$ grams masa flour, $1.5$ grams cayenne pepper, and $72$ grams of ``chili spices''.} \hbox{One can approximate Carroll Shelby's ``chili spices'' with regular chili powder} \hbox{and a pinch of paprika. Taste for seasoning when using this method.}}

    \item 2 Bags Oyster Crackers (Optional)
\end{itemize}

\instructions
Procure a crock pot with about 1 gallon capacity.
\begin{enumerate}
    \item Peel and dice the onion.
    \item Brown the grounds in a skillet over medium high heat. While browning, add about half of the large red ``chili spices'' packet and about a cup of water to help absorb the spices. When the grounds are about half done, add the diced onion.
    \item Once the grounds are browned and the onions have caramelized, add contents of skillet to crock pot.
    \item Drain the kidney beans and add them to the crock pot.
    \item Add all other canned ingredients to crock pot.
    \item Add all remaining spices to crock pot.
    \item Stir well, then cook on low for 8 hours. Stir occasionally.
    \item Optionally serve with oyster crackers and coarse salt on top. Keep in fridge for up to a week. Reheats well when covered in microwave.
\end{enumerate}
\renewcommand{\thefootnote}{\arabic{footnote}}
\spewfootnotes

\end{document}
