\documentclass{article}
\usepackage{../recipe}

%%%%%%%%%%%%%%%%%%%%%%%%%%%%%%%%%%%%

%%%%%%% Recipe Formatting %%%%%%%

% add your formatting for the first page here
\fancypagestyle{firstpage}{%
    % here is an example
    \prepcooktime{5 minutes}{1 minute 30 seconds}
    \dessert
    \serves{1}
}

%%%%%%%%%%%%%%%%%%%%%%%%%%%%%%%%%

\begin{document}
% start the recipe with the \recipe command
\recipe{Mug Snickerdoodle Cake}
% set the formatting for the first page only
\thispagestyle{firstpage}
% sign every page with name and url
\sign{Patrick Cook}{https://github.com/pdcook/Recipes}

% begin writing the recipe
\ingredients

\begin{minipage}{\textwidth}
\begin{multicols*}{2}
\begin{minipage}{\linewidth}
\ingredients[Cake]
\vspace{-1em}
\begin{itemize}
    \item 1/4 cup + 2 tbsp all purpose flour
    \item 1/4 cup milk
    \item 2 tbsp granulated sugar
    \item 2 tbsp salted butter, melted
    \item 1/2 tsp vanilla extract
    \item 1/4 tsp baking powder
    \item 1/4 tsp ground cinnamon
\end{itemize}
\end{minipage}

\columnbreak
\begin{minipage}{\linewidth}
\ingredients[Cinnamon Sugar]
\vspace{-1em}
\begin{itemize}
    \item 1 tbsp granulated sugar
    \item 1/4 tsp ground cinnamon
\end{itemize}
\end{minipage}
\end{multicols*}

\instructions
Procure two small bowls and a microwave-safe mug with a capacity of at least 12 fl.\ oz. For best results, use a cylindrical mug with straight, vertical walls.
\begin{enumerate}
    \item In a small bowl, whisk together the dry ingredients for the cake until homogeneous.
    \item Whisk in the wet ingredients until batter is smooth.
    \item In a separate bowl, thoroughly whisk together the ingredients for the cinnamon sugar.
    \item Scoop a large spoonful of the batter into the mug, then sprinkle a spoonful of the cinnamon sugar. Repeat layering until all of the batter has been transferred to the mug. Top with the remaining cinnamon sugar.
    \item Microwave uncovered on high for 1 minute to 1 minute 30 seconds. Allow to cool before serving.
\end{enumerate}
\end{minipage}

\spewfootnotes

\end{document}
