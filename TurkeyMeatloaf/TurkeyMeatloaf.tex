\documentclass{article}
\usepackage{fancyhdr}
\usepackage{multicol}
\usepackage[hidelinks]{hyperref}
\usepackage[bottom]{footmisc}
\usepackage{lastpage}

% some default formatting
\pagestyle{fancy}
\cfoot{\thepage\ of \pageref{LastPage}}
\setlength\parindent{0pt}

%%%%%%% Commands for recipes %%%%%%%

% preceed recipe with \recipe{NAME} command
\newcommand{\recipe}[1]{%
    \newpage\lhead{}\chead{#1}\rhead{}\lfoot{}\rfoot{}\section*{#1}}

% display how many servings it makes with \serves{NUMBER}
\newcommand{\serves}[1]{%
    \chead{Serves #1}}

% options for different diets: \vegetarian, \vegan, \pescetarian, \noredmeat
\newcommand{\vegetarian}{%
    \rhead{Vegetarian}}
\newcommand{\vegan}{%
    \rhead{Vegan}}
\newcommand{\pescetarian}{%
    \rhead{Pescetarian}}
\newcommand{\noredmeat}{%
    \rhead{No Red Meat}}

% options for recipe type: \breakfast, \lunch, \dinner, \snack, \dessert
\newcommand{\breakfast}{%
    \lhead{Breakfast}}
\newcommand{\lunch}{%
    \lhead{Lunch}}
\newcommand{\dinner}{%
    \lhead{Dinner}}
\newcommand{\snack}{%
    \lhead{Snack}}
\newcommand{\Dessert}{%
    \lhead{Dessert}}

% display only one of preptime or cooktime with \preptime{PREPTIME} or \cooktime{COOKTIME}
% display both preptime and cooktime with \prepcooktime{PREPTIME}{COOKTIME}
\newcommand{\preptime}[1]{%
    \lfoot{Prep time: #1}}
\newcommand{\cooktime}[1]{%
    \lfoot{Cook time: #1}}
\newcommand{\prepcooktime}[2]{%
    \lfoot{Prep time: #1\\Cook time: #2}}

% start ingredients list with \ingredients or \ingredients[HEADER]
\newcommand{\ingredients}[1][\Large\emph{Ingredients}]{%
    \emph{#1}\\}
% start instructions list with \instructions or \instructions[HEADER]
\newcommand{\instructions}[1][\Large\emph{Instructions}]{%
    \emph{#1}\\}

% temperature in farenheit with \temp{TEMPERATURE}
\newcommand{\temp}[1]{%
    $#1^\circ$F}

% sign recipe with \sign{NAME}{URL}
\newcommand{\sign}[2]{%
    \rfoot{#1\\\emph{\href{#2}{#2}}\\}}

%%%%%%%%%%%%%%%%%%%%%%%%%%%%%%%%%%%%

%%%%%%% Recipe Formatting %%%%%%%

% add your formatting for the first page here
\fancypagestyle{firstpage}{%
    % here is an example
    \noredmeat
    \prepcooktime{5 minutes}{45 minutes}
    \dinner
    \serves{4}
}

%%%%%%%%%%%%%%%%%%%%%%%%%%%%%%%%%

\begin{document}
% start the recipe with the \recipe command
\recipe{Turkey Meatloaf}
% set the formatting for the first page only
\thispagestyle{firstpage}
% sign every page with name and url
\sign{Patrick Cook}{https://github.com/pdcook/Recipes}

% begin writing the recipe
\ingredients
\begin{multicols}{2}
\ingredients[For the Loaf]
\begin{itemize}
    \item 1 lb Ground Turkey (Not Rolls)
    \item 2/3 cup Dry Breadcrumbs
    \item 1/2 Small Onion, Minced
    \item 3/4 cup Whole Milk
    \item 1 Egg, Beaten
    \item 1 tsp Soy Sauce
    \item 1/2 tsp Seasoned Pepper
\end{itemize}
\columnbreak
\ingredients[For the Sauce]
\begin{itemize}
    \item 1/3 cup Ketchup
    \item 2 tbsp Brown Sugar
    \item 1 tbsp Yellow Mustard
\end{itemize}
\end{multicols}

\instructions
Preheat oven to \temp{350}. Procure aluminum foil, a cookie sheet, a large mixing bowl, and a small mixing bowl.
\begin{enumerate}
    \item In the large mixing bowl, combine all the incredients for the loaf together. Fold gently to combine; don't overmix. The mixture should not be homogeneous.
    \item In the small mixing bowl, stir together the ingredients for the sauce until homogeneous.
    \item Place the loaf mixture onto the cookie sheet lined with aluminum foil. Shape it into a rectangle about $1\frac{1}{2}$ inches thick.
    \item Make a small ditch in the center of the loaf, just enough for all of the sauce. A ditch too deep with lead to problems when trying to cook it equally. Fill the ditch with the sauce.
    \item Bake for $40-45$ minutes, until internal temperature is \temp{160} and juices run clear.
    \item Let sit for $10$ minutes. Cut as desired and serve hot---or cold after refrigerating.
\end{enumerate}

\end{document}
