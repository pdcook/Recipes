\documentclass{article}
\usepackage{fancyhdr}
\usepackage{multicol}
\usepackage[hidelinks]{hyperref}
\usepackage[bottom]{footmisc}
\usepackage{lastpage}

% some default formatting
\pagestyle{fancy}
\cfoot{\thepage\ of \pageref{LastPage}}
\setlength\parindent{0pt}

% some custom commands for hyperlinks
\usepackage{tablefootnote}
\usepackage{dblfnote}
\hypersetup{%
    colorlinks=true,
    urlcolor=blue,
    linkcolor=blue,
    runcolor=blue,
    allcolors=blue}
\newcommand{\foothref}[2]{\href{#1}{#2}\tablefootnote{\mbox{\url{#1}}}}

\makeatletter
\newcommand{\spewfootnotes}{%
\tfn@tablefootnoteprintout%
\global\let\tfn@tablefootnoteprintout\relax%
\gdef\tfn@fnt{0}%
}
\makeatother

\makeatletter
\renewcommand\footnoterule{%
  \kern-3\p@
  \hrule\@width \textwidth
  \kern2.6\p@}
\makeatother

%%%%%%% Commands for recipes %%%%%%%

% preceed recipe with \recipe{NAME} command
\newcommand{\recipe}[1]{%
    \newpage\lhead{}\chead{#1}\rhead{}\lfoot{}\rfoot{}\section*{#1}}

% display how many servings it makes with \serves{NUMBER}
\newcommand{\serves}[1]{%
    \chead{Serves #1}}

% options for different diets: \vegetarian, \vegan, \pescetarian, \noredmeat
\newcommand{\vegetarian}{%
    \rhead{Vegetarian}}
\newcommand{\vegan}{%
    \rhead{Vegan}}
\newcommand{\pescetarian}{%
    \rhead{Pescetarian}}
\newcommand{\noredmeat}{%
    \rhead{No Red Meat}}

% options for recipe type: \breakfast, \lunch, \dinner, \snack, \dessert
\newcommand{\breakfast}{%
    \lhead{Breakfast}}
\newcommand{\lunch}{%
    \lhead{Lunch}}
\newcommand{\dinner}{%
    \lhead{Dinner}}
\newcommand{\snack}{%
    \lhead{Snack}}
\newcommand{\Dessert}{%
    \lhead{Dessert}}

% display only one of preptime or cooktime with \preptime{PREPTIME} or \cooktime{COOKTIME}
% display both preptime and cooktime with \prepcooktime{PREPTIME}{COOKTIME}
\newcommand{\preptime}[1]{%
    \lfoot{Prep time: #1}}
\newcommand{\cooktime}[1]{%
    \lfoot{Cook time: #1}}
\newcommand{\prepcooktime}[2]{%
    \lfoot{Prep time: #1\\Cook time: #2}}

% start ingredients list with \ingredients or \ingredients[HEADER]
\newcommand{\ingredients}[1][\Large\emph{Ingredients}]{%
    \emph{#1}\\}
% start instructions list with \instructions or \instructions[HEADER]
\newcommand{\instructions}[1][\Large\emph{Instructions}]{%
    \emph{#1}\\}

% temperature in farenheit with \temp{TEMPERATURE}
\newcommand{\temp}[1]{%
    $#1^\circ$F}

% sign recipe with \sign{NAME}{URL}
\newcommand{\sign}[2]{%
    \rfoot{#1\\\emph{\href{#2}{#2}}\\}}

%%%%%%%%%%%%%%%%%%%%%%%%%%%%%%%%%%%%

%%%%%%% Recipe Formatting %%%%%%%

% add your formatting for the first page here
\fancypagestyle{firstpage}{%
    % here is an example
    \prepcooktime{5 minutes}{1 minute 30 seconds}
    \breakfast
    \serves{1}
}

%%%%%%%%%%%%%%%%%%%%%%%%%%%%%%%%%

\begin{document}
% start the recipe with the \recipe command
\recipe{Mug Pancake}
% set the formatting for the first page only
\thispagestyle{firstpage}
% sign every page with name and url
\sign{Patrick Cook}{https://github.com/pdcook/Recipes}

% begin writing the recipe
\ingredients
\begin{itemize}
    \item $26$ grams salted butter
    \item $20$ grams white granulated sugar
    \item $60$ grams milk
    \item $50$ grams all purpose flour
    \item $1/2$ tsp baking powder
    \item $1/4$ tsp vanilla extract (optional)
\end{itemize}

\instructions
Procure a $12$ fl.\ oz., microwave safe coffee mug and a sufficiently precise digital scale. For best results, the measurements in this recipe must be followed very precisely.
\begin{enumerate}
    \item Add the butter to the mug and microwave until completely melted---about 40 seconds in a $1000$ W microwave.
    \item Add all other ingredients to the mug and mix until just combined. Do not over mix.
    \item Separate batter from the mug by scraping the bottom and sides with a rubber spatula.
    \item Tap mug on counter once or twice to settle the batter.
    \item Microwave uncovered for 1 minute to 1 minute 30 seconds. Do not overcook.
    \item Let cool and then serve hot, optionally with honey, maple syrup, or butter on top.
\end{enumerate}

\spewfootnotes

\end{document}
