\documentclass{article}
\usepackage{fancyhdr}
\usepackage{multicol}
\usepackage[hidelinks]{hyperref}
\usepackage[bottom]{footmisc}
\usepackage{lastpage}

% some default formatting
\pagestyle{fancy}
\cfoot{\thepage\ of \pageref{LastPage}}
\setlength\parindent{0pt}

% some custom commands for hyperlinks
\usepackage{tablefootnote}
\usepackage{dblfnote}
\hypersetup{%
    colorlinks=true,
    urlcolor=blue,
    linkcolor=blue,
    runcolor=blue,
    allcolors=blue}
\newcommand{\foothref}[2]{\href{#1}{#2}\tablefootnote{\mbox{\url{#1}}}}

\makeatletter
\newcommand{\spewfootnotes}{%
\tfn@tablefootnoteprintout%
\global\let\tfn@tablefootnoteprintout\relax%
\gdef\tfn@fnt{0}%
}
\makeatother

\makeatletter
\renewcommand\footnoterule{%
  \kern-3\p@
  \hrule\@width \textwidth
  \kern2.6\p@}
\makeatother

%%%%%%% Commands for recipes %%%%%%%

% preceed recipe with \recipe{NAME} command
\newcommand{\recipe}[1]{%
    \newpage\lhead{}\chead{#1}\rhead{}\lfoot{}\rfoot{}\section*{#1}}

% display how many servings it makes with \serves{NUMBER}
\newcommand{\serves}[1]{%
    \chead{Serves #1}}

% options for different diets: \vegetarian, \vegan, \pescetarian, \noredmeat
\newcommand{\vegetarian}{%
    \rhead{Vegetarian}}
\newcommand{\vegan}{%
    \rhead{Vegan}}
\newcommand{\pescetarian}{%
    \rhead{Pescetarian}}
\newcommand{\noredmeat}{%
    \rhead{No Red Meat}}

% options for recipe type: \breakfast, \lunch, \dinner, \snack, \dessert
\newcommand{\breakfast}{%
    \lhead{Breakfast}}
\newcommand{\lunch}{%
    \lhead{Lunch}}
\newcommand{\dinner}{%
    \lhead{Dinner}}
\newcommand{\snack}{%
    \lhead{Snack}}
\newcommand{\Dessert}{%
    \lhead{Dessert}}

% display only one of preptime or cooktime with \preptime{PREPTIME} or \cooktime{COOKTIME}
% display both preptime and cooktime with \prepcooktime{PREPTIME}{COOKTIME}
\newcommand{\preptime}[1]{%
    \lfoot{Prep time: #1}}
\newcommand{\cooktime}[1]{%
    \lfoot{Cook time: #1}}
\newcommand{\prepcooktime}[2]{%
    \lfoot{Prep time: #1\\Cook time: #2}}

% start ingredients list with \ingredients or \ingredients[HEADER]
\newcommand{\ingredients}[1][\Large\emph{Ingredients}]{%
    \emph{#1}\\}
% start instructions list with \instructions or \instructions[HEADER]
\newcommand{\instructions}[1][\Large\emph{Instructions}]{%
    \emph{#1}\\}

% temperature in farenheit with \temp{TEMPERATURE}
\newcommand{\temp}[1]{%
    $#1^\circ$F}

% sign recipe with \sign{NAME}{URL}
\newcommand{\sign}[2]{%
    \rfoot{#1\\\emph{\href{#2}{#2}}\\}}

%%%%%%%%%%%%%%%%%%%%%%%%%%%%%%%%%%%%

%%%%%%% Recipe Formatting %%%%%%%

% add your formatting for the first page here
\fancypagestyle{firstpage}{%
    % here is an example
    \noredmeat
    \prepcooktime{30 minutes}{7 to 9 hours}
    \lunch
    \serves{6}
}

%%%%%%%%%%%%%%%%%%%%%%%%%%%%%%%%%

\begin{document}
% start the recipe with the \recipe command
\recipe{Buffalo Chicken Sandwiches}
% set the formatting for the first page only
\thispagestyle{firstpage}
% sign every page with name and url
\sign{Patrick Cook}{https://github.com/pdcook/Recipes}

% begin writing the recipe
\ingredients
\begin{minipage}{\textwidth}
\begin{multicols*}{2}
\ingredients[Buffalo Chicken]
\begin{itemize}
    \item $1$ bottle mild or spicy buffalo wing sauce (such as \foothref{https://www.franksredhot.com/en-us/products/franks-redhot-buffalo-wings-sauce}{Frank's})
    \item $3$ to $4$ lbs boneless, skinless chicken breast
    \item $1$ small white onion, diced or chopped
    \item $1$ to $2$ green bell peppers, diced or chopped
    \item $2$ to $4$ cloves garlic, finely minced or crushed
    \item $1$ package dry ranch salad dressing mix
    \item $2$ tbsp salted butter
\end{itemize}
\columnbreak
\ingredients[Sandwich]
\begin{itemize}
    \item Hoagie Rolls, split lengthwise
    \item Sliced Swiss, sliced provolone, or freshly grated Parmesan
    \item Any additional desired sandwich toppings
\end{itemize}
\end{multicols*}
\end{minipage}
\vspace{1em}

\instructions
Procure a crock pot with at least $1$ gallon capacity.
\begin{enumerate}
    \item Chop or dice the onion and green bell peppers to desired size and finely mince or crush the garlic.
    \item Combine all the ingredients for the buffalo chicken in the crock pot---except for the butter.
    \item Cook on low for $7$ to $9$ hours, until internal temperature of the chicken reaches \temp{165}.
    \item Once cooked, shred the chicken. This can be done either with two forks or a large spaghetti spoon.
    \item Add the butter to the shredded chicken and mix gently until completely melted.
    \item To serve, toast hoagie rolls. Optionally, melt cheese and pepper onto rolls while toasting. Pile buffalo chicken on rolls and add any additional toppings.
    \item If desired, keep buffalo chicken sealed in fridge for up to a week. Reheat covered in microwave, prepare sandwich as before.
\end{enumerate}

\spewfootnotes

\end{document}
