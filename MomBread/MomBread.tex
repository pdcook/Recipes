
\RequirePackage{../recipe}
%%%%%%%%%%%%%%%%%%%%%%%%%%%%%%%%%%%%

%%%%%%% Recipe Formatting %%%%%%%

% add your formatting for the first page here
\fancypagestyle{firstpage}{%
    % here is an example
    \prepcooktime{1 hour 30 minutes}{40 minutes}
    \snack
    \serves{Many}
}

%%%%%%%%%%%%%%%%%%%%%%%%%%%%%%%%%

\begin{document}
% start the recipe with the \recipe command
\recipe{Mom Bread}
% set the formatting for the first page only
\thispagestyle{firstpage}
% sign every page with name and url
\sign{Jessica Ullom-Minnich and Patrick Cook}{https://github.com/pdcook/Recipes}

% begin writing the recipe
\ingredients
\renewcommand{\thefootnote}{\alph{footnote}}

\begin{minipage}{\textwidth}
\begin{multicols*}{2}
\begin{minipage}{\linewidth}
\ingredients[Dry]
\vspace{-1em}
\begin{itemize}
    \item $2-4$ cups all purpose flour
    \item $2$ cups whole wheat flour\footnotemark[1]
    \item $1$ cup plain rolled oats
    \item $1$ cup rye flour\footnotemark[2]
    \item $1$ cup dry milk powder
    \item $1/4$ cup wheat gluten
    \item $1/4$ cup wheat germ\footnotemark[3]
    \item $1/4$ cup wheat bran\footnotemark[3]
    \item $1/4$ cup oat bran\footnotemark[3]
    \item $1/2$ tbsp active dry yeast
    \item $1$ tbsp salt
\end{itemize}
\end{minipage}

\columnbreak
\begin{minipage}{\linewidth}
\ingredients[Wet]
\vspace{-1em}
\begin{itemize}
    \item $2~\frac{1}{4}$ cups water
    \item $1/4$ cup honey\footnotemark[4]
    \item $1/4$ cup salted butter
    \item $1/2$ tbsp active dry yeast
\end{itemize}
\ingredients[Miscellaneous]
\vspace{-1em}
\begin{itemize}
    \item 1 tbsp salted butter
    \item nonstick cooking spray
    \item a pinch coarse salt (optional)
\end{itemize}
\end{minipage}
\end{multicols*}
\end{minipage}
\setcounter{footnote}{0}
\vspace{1em}

\instructions
Procure a stand mixer, $8~\mathrm{in.}\times4~\mathrm{in.}$ bread pan, and a wire cooling rack. Make space on a clean surface to knead dough.
\begin{enumerate}
    \item From the wet ingredients, combine the water, honey, and butter---\textbf{not} the yeast---in a microwave safe bowl.
    \item Microwave the mixture uncovered for $2$ minutes on high.\footnotemark[5]
    \item Add the $1/2$ tbsp of yeast to the warmed mixture.
    \item In a stand mixer set to the slowest setting, combine all of the dry ingredients except for the all purpose flour.
    \item Once the dry ingredients are thoroughly mixed, add the wet mixture and keep the stand mixer running.
    \item To the running stand mixer, add $1/4$ cup of all purpose flour at a time until the dough separates from the sides of the bowl and forms a large, slightly sticky ball.
    \item Pour the dough onto a well floured (with all purpose flour) kneading surface and set aside stand mixer---do not rinse or clean yet. Flour hands and knead until dough bounces when depressed. Add more all purpose flour as necessary to keep it from sticking.
    \item Spray the bowl of the stand mixer with nonstick cooking spray. Place dough inside and cover with kitchen towel. Let rise at room temperature until doubled in size, about $30$ minutes.
    \item Once risen, remove dough to well floured surface and divide into two equal balls.
    \item Using a rolling pin, separately roll each ball into a $1/2~\mathrm{in.}$ thick rectangle with the long side roughly twice as long as the short side.
    \item Starting from one of the short ends of the rectangle, tightly roll the dough into a cylinder. Pinch closed the seam at the opposite short end of the rectangle, as well as the ends of the cylinder. Repeat this and previous step for second dough ball.\footnotemark[6]
    \item Spray bread pan with nonstick cooking spray. Place dough in prepared bread pan.
    \item Cover with kitchen towel and allow to rise at room temperature for $30$ minutes or until doubled in size.
    \item Once risen, bake in oven at \temp{350} for about $40$ minutes or until evenly browned. When done, the loaf should sound hollow when tapped and a thermometer inserted reads \temp{200}.
    \item Immediately remove the loaf from the bread pan and place onto a wire cooling rack. Immediately spread butter over top and coarse salt if desired.
    \item Let cool completely before slicing. Store in airtight container at room temperature.
\end{enumerate}
\renewcommand{\thefootnote}{\alph{footnote}}
\footnotetext[1]{If necessary, can be substituted for all purpose flour.}
\footnotetext[2]{If necessary, can be substituted for whole wheat flour (preferred) or all purpose flour.}
\footnotetext[3]{If necessary, wheat germ, wheat bran, and oat bran can be substituted for one another.}
\footnotetext[4]{Can be substituted with molasses or sugar.}
\footnotetext[5]{Alternatively, heat on stove until butter is just melted (preferred) or just use hot water.}
\footnotetext[6]{If desired, wrap prepared dough cylinder in plastic wrap and freeze for up to a couple months. To use, remove from freezer, unwrap, and allow to thaw covered at room temperature for $6$ hours or until doubled in size.}
\renewcommand{\thefootnote}{\arabic{footnote}}

\spewfootnotes
\end{document}
